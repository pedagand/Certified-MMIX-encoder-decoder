\documentclass {article}

%% ** Packages

\usepackage{natbib}
\usepackage[utf8]{inputenc}
\usepackage[french]{babel}
\usepackage{amsthm, amssymb, amsmath}
\usepackage{hyperref}
\usepackage{mathpartir}

%% ** OCaml listings

\usepackage{xcolor}
\usepackage{listings}
\lstset{
  language=[Objective]Caml,
  %% emph={[2]Nil,Cons,FZe,FSu,ze,su,Ze,Su},
  %% emphstyle={[2]\Constructor},
  %% emph={[3]lookup,failwith},
  %% emphstyle={[3]\Function},
  %% emph={[4]vec,fin,nat,list,tree,completeTree},
  %% emphstyle={[4]\Canonical},
  morecomment=[s]{(*}{*)},
  rangeprefix=\(\*\=,
  rangesuffix=\ \*\),
  includerangemarker=false,
  extendedchars=\true,
  inputencoding=utf8,
  showspaces=false,
  showstringspaces=false,
  showtabs=false,
  basicstyle=\ttfamily\small,
  framesep=4mm,
  moredelim=*[s][\itshape]{(*}{*)},
  moredelim=[is][\textcolor{darkgray}]{§}{§},
  escapechar=°,
  keywordstyle=\color[rgb]{0.627451, 0.125490, 0.941176},
  stringstyle=\color[rgb]{0.545098, 0.278431, 0.364706},
  commentstyle=\color[rgb]{0.698039, 0.133333, 0.133333},
  numberstyle=\color[rgb]{0.372549, 0.619608, 0.627451},
  boxpos=t,
  literate= {'a}{{$\alpha$}}1%
  {->}{{${\to}$}}2
  {*}{{${\times}$}}1
  {::}{{${:\::}$}}1
}

\newcommand{\codefrom}[3]
           {\lstinputlisting[linerange={#3}-End]{../#1/#2.v}}

\newcommand{\codefromOcaml}[3]
           {\lstinputlisting[linerange={#3}-End]{../#1/#2.ml}}
           
%% ** Theorem styles 
           
           
\newtheorem{theorem}{Théorème}
\newtheorem{proposition}{Proposition}
\newtheorem{lemma}{Lemme}

\theoremstyle{definition}
\newtheorem{definition}{Définition}
\newtheorem{example}{Exemple}

\theoremstyle{remark}
\newtheorem{remark}{Remarque}
\newtheorem{para}{} 

%% ** commands

\newcommand{\todo}[1]{\textcolor{red}{#1}}
\newcommand{\attention}[1]{\textcolor{orange}{#1}}
\newcommand{\question}[1]{\textcolor{green}{#1}}
\newcommand{\etc}{\textit{etc.}}

\newenvironment{bnf}
               {\[\begin{array}{lcl@{\qquad}r}}
               {\end{array}\]}
               
\newcommand{\fun}[1]{\lstinline!#1!}
%% ** Title

\title{Assembleur x86 certifié}
\author{Roman Delgado}
\date{}


\begin{document}

\maketitle


%% ** Abstract 

\vfill
\setcounter{tocdepth}{2}
\tableofcontents
\vfill


\section{Introduction}



Un assembleur est un programme permettant de traduire des instructions lisibles par un être humain vers
une de leur représentation binaire associée afin d'être éxécutées par le processeur. Contrairement à un compilateur
qui effectue lui aussi un travail de traduction, l'assembleur se focalise uniquement sur l'analyse syntaxique
et non sur l'aspect sémantique.
Il existe de nombreux langages 
assembleurs pour citer les plus connus \fun{x86}, \fun{arm}.
Le programme KVM
utilise nottament une fonction de décodage des instructions.
Cependant certains problèmes on été relevés au niveau de celle ci.
Même si ces problèmes on été résolus on ne peut à l'heure actuelle pas garantir que d'autre bugs ne seront pas
découverts dans les années à venir ~\citep{Amit:2015:VCV:2815400.2815420}.

Prenons l'exemple d'un autre logiciel dont l'utilisation est très répandue : \fun{gcc}. C'est l'un des compilateurs
du langage C le plus utilisé à l'heure actuelle et pourtant il existe encore des
problèmes avec celui- ci. C'est ce qu'a démontré Xavier Leroy en concevant un compilateur certifié pour le langage C \fun{Compcert}
~\citep{Leroy14thecompcert}. 
Après avoir testé les deux compilateurs certains bugs avec \fun{gcc} on été relevés et aucuns avec \fun{compcert}.


Mais comment garantir qu'un programme s'exécute correctement? En quoi consiste la réalisation d'un logiciel certifié?
Ces questions sont des questions centrales depuis le développement de l'informatique, et de nombreux travaux ont été
réalisés sur ce sujet. Les méthodes pour répondre à ces problèmatiques sont par exemple le model checking, les assistants
de preuve ect...


Les assistants de preuves permettent de créer des programmes sur lesquels on énonce des propriétés que l'on prouve par
la suite avec celui- ci. L'assistant de preuve permet d'assister comme son nom l'indique la rédaction de preuves mais
aussi de vérifier automatiquement leur validitée.

La difficulté lors de la réalisation d'un assembleur ne se situe pas en hauteur; en effet le nombre de
fonctions nécéssaires n'est pas très conséquent. Par contre la complexité en largeur
de ce type de programme est très grande. Cela est du au nombre d'instructions ainsi qu'à leur syntaxe qui selon
la spécification du langage peut être particulièrement riche.
Les assistants de preuves se prêtent très bien à ce genre de problème car l'on peut prouver des
propriétés sur des ensembles (dans notre cas des instructions) et réaliser une seule preuve pour
un ensemble donné. Pour ce projet l'assistant de preuve utilisé sera le langage \fun{Coq} ~\citep{bertot:coqart}.

Une motivation supplémentaire est la continuation des travaux initiés par Compcert.
Le compilateur \fun{Compcert} génère du langage assembleur x86, mais il n'existe pas
de programme permettant de garantir sa traduction en bits.
C'est donc le but de ce projet qui est la réalisation d'un assembleur x86 certifié.

Afin de mener à bien ce projet et vue la complexité de la spécification de x86,
nous allons tout d'abord réaliser un assembleur préliminaire basé sur le langage MMIX.



%%-----Contributions-------
\subsection{Contributions}
Voici une liste des réalisations apportées pour mener à bien ce projet:
\begin{enumerate}
\item Ecole d'hiver de Coq à sophia Antipolis de Yves Bertot.
\item Réalisation de la conversion d'entiers vers une représentation binaire \ref{Section Conversion N liste B}.
\item Première représentation des données pour MMIX (seulement 2 instructions).
\item Travail sur la réflexion afin d'implémenter les théorèmes sur la liste d'associations \ref{Lemmeslol}.
\item Première version des fonctions \fun{encode} et \fun{decode} \ref{Encode Decode}
\item Ajout de l'ensemble des instructions de MMIX dans le programme.
\item Tentative de remaniement de la représentation des données afin de factoriser les preuves \ref{LemmesEncode}.
\item Lectures sur le sujet des librairies de parser pour l'appliquer à MMIX ~\citep{Swierstra2008CombinatorPA}.
\end{enumerate}



%%--------------------------------------MMIX et représentation des données-----------------
\section{MMIX et représentation des données}
\label{partieMMIX}

Le langage d'assembleur MMIX à été crée par Donald Knuth en 1999.
A l'heure actuelle il n'existe aucun processeur utilisant ce jeu d'instructions, il
à été crée et utilisé dans un but pédagogique.



%%-------------Definition---------------
\subsection{Définition MMIX}

Dans cette section il sera question de définir la syntaxe des instructions MMIX.
MMIX est un jeu d'instructions RISK ce qui signifie que la taille des instructions
est fixe. Une instruction a une taille de 32 bits, voici une première représentation:

\begin{tabular}{|c|c|c|c|}
  \hline
  Etiquette & Operande 1 & Operande 2 & Operande 3\\
  \hline
  0-7 bits & 8-15 bits & 16-23 bits & 24-31  \\
  \hline
  TAG & X & Y & Z \\
  \hline
\end{tabular}
\remark{La dernière ligne du tableau permet simplement d'associer à chaque partie
  de l'instruction un nom de variable afin d'alléger le texte dans la suite du rapport}\\
Les operandes peuvent être à la fois un immediat ou un registre.
Une des spécificité de MMIX est la possibilité d'obtenir des operandes
de plus grande taille en en utilisant un nombre réduit:\\
\\
\begin{tabular}{|c|c|c|}
  \hline 
  TAG & X & YZ \\
  \hline
\end{tabular}\\
\begin{tabular}{|c|c|}
  \hline 
  TAG & XYZ \\
  \hline
\end{tabular}




%%-------------Representation---------------
\subsection{Représentation des données}
\label{representation des donnees}

Dans un premier temps j'ai fait le choix d'utiliser une représentation des données très exhaustive sans
chercher à la factoriser. Ce choix est du à mon apprentissage de Coq, il était préférable de ne pas
rajouter une difficulté supplémentaire lors de la réalisation des preuves.
Un autre choix à été de stocker l'information sur la syntaxe de l'instruction grâce au type des étiquettes.
Cela permettra de ne pas mettre cette information dans la liste d'associations, nous y reviendrons dans la
section \ref{partieOpcode}.

\codefrom{src}{ast_instructions}{tag}

Chaque type d'étiquette sera associé à une syntaxe d'instructions par exemple le type ``tag\_ter\_normal''
regroupe les tags ne pouvant être utilisés que pour les instructions dont les trois
opérandes sont toutes des registres.

\codefrom{src}{ast_instructions}{tag_definition}
...

\remark{Pour un langage assembleur tel que x86 il ne sera pas possible de définir les instructions de façon exhaustive il faudra
  utiliser une autre représentation des données.}\\ 

Maintenant nous voulons définir un type de données qui puisse représenter une instruction.
Etant donné que nous avons dissocié les différents types de tags en fonction de la syntaxe de l'instruction,
nous allons poursuivre de manière analogue.
Voici un exemple pour les instructions dont les 3 opérandes représentent des registres.

\codefrom{src}{ast_instructions}{instruction_tern_n}

L'utilisation du type enregistrement convient parfaitement pour représenter les instructions car il permet de stocker
l'ensemble des informations de celles- ci.

Nous obtenons donc le type suivant pour les instructions:
\codefrom{src}{ast_instructions}{instruction}




%%-------------Egalité---------------
\subsection{Egalité entre deux éléments d'un type de donnée}
Avant de passer à la section suivante attardons nous sur l'égalité de deux éléments de même type.
\remark{Dans cette section nous nous attarderons uniquement sur les types de données inductifs}\\
Si l'on souhaite définir une égalité pour le type de données suivant :
\codefrom{rapport}{definitions}{example_type}

Il va nous falloir créer une fonction exhaustive énumérant l'ensemble des cas pour déterminer
si les deux élements sont égaux ou non. Ce qui nous donne pour l'exemple précédent la fonction suivante:

\codefrom{rapport}{definitions}{equal}

Ces fonctions sont très fastidieuses à rédiger et l'on constate déja la taille de celle ci
pour un type de données composé de deux éléments.
Si nous souhaitions écrire cette fonction le type \fun{tag} dans
la section \ref{representation des donnees} la perte de temps serait considérable.
Dans Coq il existe une commande qui va nous permettre de générer cette fonction
automatiquement, la voici:

\codefrom{rapport}{definitions}{SchemeEqual}



%%-------------Lemmes---------------
\subsection{Lemmes}
\label{LemmesAST}
Définisons notre premier lemme:
\codefrom{src}{ast_instructions}{tag_beq_different}
Ici on désire montrer que si l'on prossède une preuve que deux étiquettes sont équivalentes
à l'aide de notre fonction booléene alors on peut déduire une preuve
de cette égalité dans le monde propositionnel.








%%----------------------------------------Section Conversion N liste B -----------------------------
\section{Conversion N - bits}
\label{Section Conversion N liste B}

Cette section s'attarde sur la représentation que nous donnerons à nos instructions
sous forme binaire. Il sera aussi également question de conversion des entiers vers cette
représentation étant donné que les immédiats ainsi que les registres sont stockés sous forme d'entier
pour les instructions.



%------------repres bits ----------------
\subsection{Représentation des bits}


Il nous faut déterminer un type de données pour représenter
les instructions sous forme binaires.
Notre type de données doit pouvoir nous permettre de :
\begin{enumerate}
\item Ne pas avoir de restriction au niveau de la taille car lors du parsing d'un fichier binaire,
  on souhaite générer un flux de bits puis le découper au fur et à mesure du décodage.
\item Pouvoir encoder un entier sur un nombre de bits quelconque.
\item Il doit être simple car il est préférable que les preuves qui sont le coeur de ce projet restent
  plus aisées à mettre en place.
\end{enumerate}

Il existe dans la librairie standard de Coq une représentation binaire pour les entiers naturels.
Le problème de celle- ci est que l'on ne peut pas représenter un entier naturel sur un nombre de bits différents
que celui nécessaire à son encodage. Nous ne pouvons pas nous permettre d'avoir une taille variable en
fonction de la valeur encodée et cela contredirait le second pré-requis.

La structure de données par excellence qui satisfait l'ensemble de nos pré-requis est la liste de booléens.
Un booléens représentant un bit et la liste le flux.

\codefrom{src}{ast_instructions}{binary_instruction}




%-------------Fcts de conversion --------------
\subsection{Fonctions de conversion}
\label{Fonctions de conversion}
Maintenant que nous avons fixé une représentation pour nos suites de bits, commençons par définir
une fonction de conversion d'une liste de booléens vers un entier naturel:

\codefrom{src}{binary}{bitn}

Commençons par prêter attention à cette fonction car elle est une simple implémentation
d'une formule permettant de déterminer la valeur d'un nombre sous forme binaire
en forme décimale. Passons maintenant à la fonction effectuant l'opération inverse:

\codefrom{src}{binary}{nbit}


Ici cette fonction possède plusieurs subtilités d'implémentation dues
à une restriction du langage Coq.

La première chose que l'on peut remarquer se situe au niveau des arguments de
la fonction. En effet nous souhaitons simplement encoder un entier naturel vers
une liste de booléens et pourtant nous avons deux arguments.
En réalité nous souhaiterions pouvoir écrire notre fonction comme ceci
(implémentation en Ocaml):
\codefromOcaml{rapport}{example}{nbit}

Cependant cette implémentation est impossible avec Coq car celui- ci utilise
des opérateurs de points fixes et dans ce dernier exemple l'appel récursif ne s'éffectue
pas avec le prédécesseur de \fun{k}  mais avec le
résultat de \fun{k} par deux. Lors des preuves si l'on fait des récursions sans faire appel
au prédécesseur nous ne pourrons pas faire de raisonement par induction. 

\remark{Dans notre cas cette fonction nous convient parfaitement car nous souhaitons
  obtenir une liste de booléens de taille déterminée. Cependant cela nous permet d'illustrer
  la façon dont Coq définit la récursion}

On pourra aussi noter le type de retour de la fonction.
On utilise une monade bien connue des programmeurs fonctionnels au travers du type
\fun{option}. Cela nous permet de
dire que la fonction peut échouer sans pour autant devoir renvoyer une valeur par
défaut ou lever une exception (en d'autres termes que la fonction est partielle). Ici le cas ou la fonction peut échouer est lorsque
l'entier naturel que l'on souhaite encoder nécessite plus de bits que l'entier n alors
on ne peut pas réaliser la transformation.





%%--------------Lemmes-----------
\subsection{Lemmes}
\label{lemmes bits}

Il nous faut maintenant nous demander quelles sont les propriétés qui nous interressent sur
les deux fonctions que nous venons de définir dans la section \ref{Fonctions de conversion}.

Une propriété qui s'avèrera surement utile par la suite est le Lemme suivant:

\codefrom{src}{binary}{size_n_bit}

Ce lemme permet de vérifier que la taille d'une instruction produite par
la fonction \fun{n_bit} est bien de la taille de l'argument n.


Une des informations importante à retenir de ce théorème est sa structure.
En effet comme nous avons vu précédement \ref{Fonctions de conversion} la
fonction \fun{n_bit} peut échouer, il nous faut donc vérifier cette propriété
uniquement lorsque celle- ci termine correctement. C'est pour cette raison
que nous devons utiliser l'implication pour ne considérer que ces cas précis. \\

Il est maintenant question de définir les deux théorèmes les plus importants de
cette section. Notre but pour l'ensemble des fonctions que l'on définit est de
montrer qu'elles sont inverses.
car elles seront utilisées dans l'encodage et le décodage des instructions \ref{Encode Decode}
Voici donc le premier théorème que nous voulons prouver:

\codefrom{src}{binary}{nbitn}

Comme pour le lemme précédent ce théorème suit la même structure avec une implication en raison des effets
de bords.
Pour le théorème suivant nous avons besoin d'émettre une hypothèse supplémentaire sur l'argument
\fun{n}. Si nous n'effectuons pas cette assertion supplémentaire le
théorème n'a pas de sens et n'est par conséquent pas prouvable.

\codefrom{src}{binary}{bitnbit}





%%---------------------------Conversion Opcode list B---------------------------
\section{Conversion Etiquette - N}
\label{partieOpcode}

Une instruction possède une étiquette ainsi que des opérandes (\ref{partieMMIX}).
Pour réussir à encoder ou décoder une instruction il nous faut donc d'abord
réussir à convertir une étiquette en un entier \ensuremath{\mathbb{N}}.
Pourquoi cette conversion s'effectue-t-elle vers l'ensemble des entiers naturels
et non vers les listes de \ensuremath{\mathbb{B}}? Tout simplement pour éviter
de les rédiger statiquement. Nous utiliserons les fonctions définies dans
la section précédente \ref{Section Conversion N liste B} lorsque nous souhaiterons transformer nos entiers naturels.

%%---------Structure de donnée----------------
\subsection{Structure de donnée}
\label{Structure de donnee}
Dans la spécification du langage MMIX (section \ref{partieMMIX})
une étiquette est associée à un unique opcode. Nous souhaitons donc
utiliser un type de données permettant de réaliser l'association de ces deux éléments,
ce qui nous conduit à utiliser le produit cartésien suivant:
\codefrom{src}{association_list}{assoc}

Etant donné que le nombre
d'instructions est de 256, l'utilisation des listes est envisageable.
Un argument supplémentaire est que la liste
est un type inductif qui nous permettra de raisonner plus aisément
lors des preuves étant donné sa relative simplicité. En effet on pourrait utiliser
un dictionnaire pour réduire la complexité, cependant l'utilisation de ce 
type de données rendrait les preuves plus compliquées.
Nous obtenons donc le type suivant pour représenter notre collection d'associations:

\codefrom{src}{association_list}{tag_opcode_assoc}

\remark{L'utilisation des listes à l'instar des
  dictionnaires permet également de n'utiliser qu'une seule liste d'associations qui
  permettra d'encoder et de décoder contrairement à ce dernier.
}

\remark{J'ai fait le choix de ne pas me reposer sur la liste d'associations pour stocker des informations sur le décodage
  des instructions. En effet lors du décodage avec une telle liste d'associations nous n'aurons pas d'information
  sur la syntaxe de celle- ci (type des opérandes et nombre d'arguments). Cependant cette information sera obtenue ici
  grâce au type de l'étiquette qui nous indique la syntaxe. Ce choix sera critiqué dans la section \ref{Encode Decode}}

%%-----------lookup et lookdown------
\subsection{Fonctions de recherche}

Il est désormais question de réaliser deux fonctions permettant la recherche
au sein d'une liste d'associations telles  que celle définies dans la section précédente.
\ref{Structure de donnee}.

\codefrom{src}{association_list}{lookup}

Encore une fois l'utilisation de la monade option est indispensable ici car
on ne peut pas garantir que pour une liste donnée nous trouverons l'entier
qui lui est associé. Notre travail sera de démontrer que cette fonction n'échoue pas
sur une liste contenant l'ensemble des associations. Nous y reviendrons dans la section
\ref{Lemmeslol}
Voici la fonction effectuant la recherche dans le sens inverse en suivant le même principe:

\codefrom{src}{association_list}{lookdown}



%%----------Lemmes----------
\subsection{Lemmes}
\label{Lemmeslol}

Avant de commencer toute preuve il nous faut définir de manière statique une liste
contenant l'ensemble des associations entre les \fun{tag} et leur entier naturel associé.


\codefrom{src}{association_list}{encdec}
.....\\ \\

Tous les théorèmes que nous définirons se baserons sur la liste \fun{encdec}. \\
Nous avons besoin de vérifier que la liste contient bien l'ensemble des associations,
il existe deux manières de faire. Une première façon qui consiste à faire une analyse de cas
au sein de la preuve sur le type \fun{tag}.
Une autre méthode consiste à vérifier de façon calculatoire cette propriété, puis de montrer
une équivalence entre ce calcul et sa proposition associée. 
Cette opération en Coq peut être réalisée avec l'aide du type \fun{reflect}:

\codefrom{rapport}{definitions}{reflect}

Pour mieux comprendre en quoi ce type va nous être utile, il faut comprendre son utilisation.
Le type ``reflect'' est composé de deux constructeurs, si l'on possède un élément de type
``reflect P true'' alors cela signifie que notre élément est le constructeur ``ReflectT''
et celui- ci possède une preuve du prédicat ``P''. Le raisonnement est le même pour ``reflect P false''
sauf que dans ce cas on pourra obtenir une preuve de non ``P''. Pour vulgariser ce
principe on peut dire qu' avoir une preuve de ``reflect P b'' implique que la preuve
de P est équivalente à ``b = true''. Ce principe est illustré par le théorème suivant:

\codefrom{rapport}{definitions}{reflect_iff}


Mettons de côté le type ``reflect'' et définissons maintenant une fonction
pour vérifier calculatoirement un prédicat sur l'ensemble des tags:

\codefrom{src}{association_list}{forall_tag}

L'ensemble des fonctions ``forall\_tag\_...'' suivent le même schéma que la fonction suivante:

\codefrom{src}{association_list}{forall_tag_uno}

Tout comme nous l'avions fait lorsque nous avions défini la fonction d'égalité sur les tags \ref{LemmesAST},
nous souhaitons pouvoir passer du monde calculatoire au monde propositionnel:

\codefrom{src}{association_list}{helpBefore1}
\codefrom{src}{association_list}{helpBefore2}

Afin de définir des prédicats pour la fonction \fun{forall\_tag} il nous faut définir un
équivalent au connecteur propositionnel \ensuremath{\rightarrow} :

\codefrom{src}{association_list}{imply}

L'obectif final de cette section est de démontrer les théorèmes suivants (dont la structure est analogue aux lemmes prouvés
en dans la section précédente \ref{lemmes bits}):

\codefrom{src}{association_list}{lookup_lookdown}
\codefrom{src}{association_list}{lookdown_lookup}


Si nous souhaitons utiliser la réflection il nous faut définir des fonctions reflétant
ces théorèmes dans l'ensemble des booléens. La fonction pour le théorème ``lookup\_lookdown''
est la suivante:

\codefrom{src}{association_list}{lookup_encdec}
``eq\_mtag'' et ``eq\_mtag'' sont simplement des fonctions d'égalité sur les types ``option tag'' et ``option nat''.
L'intérêt d'avoir utilisé reflect pour réaliser nos preuves peut se voir immédiatement quant à la taille
du script de preuves du théorème ``lookup\_lookdown'' que voici:

\codefrom{src}{association_list}{lookup_lookdown}

\codefrom{src}{association_list}{lookup_lookdown_script}

Certains lemmes ont ici été passés volontairement sous silence car l'objectif n'est pas
de comprendre reflect dans les moindres détails mais simplement de comprendre son intérêt.



%%------------------------Encode Decode----------------------
\section{Encode et decode}
\label{Encode Decode}

Nous arrivons à la dernière partie du programme, l'enjeu de cette section sera de combiner l'ensemble des fonctions
déjà définies pour obtenir un assembleur fonctionnel.

%%---------Fcts prélminaires -------
\subsection{Fonctions prélminiaires}

Afin de réaliser les fonctions d'encodage et de décodage de nos instructions il nous manque encore quelques opérations.
Les premières fonctions que nous allons définir sont deux fonctions permettant de transformer une opérande en liste de booleens.

\codefrom{src}{encode}{operand_to_bin}

Cette fonction encode des operandes sur 8 bits, il nous faut une fonction analogue pour les opérandes sur 16 bits.

\codefrom{src}{encode}{operand_to_bin_double}

Définisson quelques lemmes sur ces nouvelles fonctions.
\codefrom{src}{encode}{operand_to_bin_hypothesis_reg}
\codefrom{src}{encode}{operand_to_bin_hypothesis_imm}
\codefrom{src}{encode}{operand_to_bin_size}


Pour la fonction de décodage nous aurons besoin d'une fonction permettant de découper
une liste de booléens en plusieurs sous listes. Voici la fonction qui effectura cette operation:

\codefrom{src}{encode}{get_first_n_bit}

La fonction retourne un couple de deux listes, la première correspond aux \fun{size} premiers
bits de la liste et la seconde au reste  de la liste. Le fait de retourner le reste de la liste
nous permettra de découper celle- ci à nouveau.





\subsection{Monade bind}

La fonction encode utilise les fonctions que nous avons définies dans les sections précédentes qui sont généralement
des fonctions partielles. Lorsque l'on souhaite utiliser leur valeur de retour nous sommes obligés de faire un
filtrage de la forme :

\codefrom{rapport}{definitions}{pattern_matching}

Cette construction est très lourde dans la définition des fonctions nous allons donc créer une monade
afin de faciliter l'utilisation de ces fonctions partielles:

\codefrom{src}{encode}{monade}

Nous avons maintenant une fonction bind qui nous permet d'automatiser ce filtrage. Coq nous permet de
définir de nouvelles notations comme celle ci:

\codefrom{src}{encode}{notation}




%%--------Encode decode-----
\subsection{Implémentation des fonctions Encode et Decode}

Commençons par la définition de la fonction encode. D'après le type instruction \ref{partieMMIX}
chaque instruction s'encode différement, nous allons devoir définir une fonction comme celle- ci
pour chaque type d'instruction:

\codefrom{src}{encode}{encode_t_n}

Ce qui nous permet de définir la fonction encode suivante:

\codefrom{src}{encode}{encode}

La fonction de décodage quant à elle n'est pas composée de plusieurs fonctions car l'entrée
étant une liste de booléens.
Voici une version partielle du code de décode:

\codefrom{src}{encode}{decode}

Ici la définition que nous avons utilisée pour les tags s'avère utile car elle nous permet
une fois récupérée dans la liste d'associations de déterminer immédiatement quel type
d'instruction il nous faut créer.





%%----Lemmes----------
\subsection{Lemmes}
\label{LemmesEncode}
De même que pour la définition d'encode nous allons diviser le travail pour chaque
type d'instructions.

\codefrom{src}{encodeProof}{encode_decode_t_n}

Le problème de la représentation actuelle des données fait que nous sommes obligés de réaliser une preuve
similaire mais non triviale pour chaque type d'instruction. Cela crée énormément de redondance dans les preuves
et cela n'est jamais bon signe. Nous verrons dans la conclusion \ref{conclusion} les solutions envisageables afin
de surmonter ce problème.
Cela nous permet finalement d'obtenir la preuve qu'une instruction encoder par notre assembleur puis decoder par celui
ci est identique. On a donc une preuve que la syntaxe est belle et bien préservée.

Définissons le théorème opposé à celui- ci à savoir :

\codefrom{src}{decodeProof}{decode_encode}

Tout comme pour la preuve d'encode ici nous devons effectuer la preuve pour chaque type d'instructions
ce qui rend une nouvelle fois la preuve très verbeuse et redondante. 




%%-------Theoremes finaux------
\subsection{Theorèmes finaux}

A l'heure actuelle nous avons une preuve que notre assembleur suit bien la specification
pour une seule instruction. Cependant un assembleur complet travaille sur un ensemble d'instructions
ou bien un flux de bits. Dans notre cas nous allons définir une fonction décodant une liste d'instructions et qui
retournera une liste de booléens de taille \fun{32 * length(liste\_instructions)}.
Pour ce faire il nous faut une fonction préalable :

\codefrom{src}{encode}{concat_listes_32}

A noter le test sur la taille de la liste
qui nous permettra d'obtenir une hypothèse supplémentaire lors de nos preuves.
Etant donné que la fonction encode utilise la monade option nous sommes obligés de définir une fonction
traverse permettant de transformer une liste de type \fun{list (option A)} en une liste de type \fun{list A}.

\codefrom{src}{encode}{traverse}

Voici la définition de la fonction encodant une liste d'insructions (pour la lisibilité celle- ci est
divisée en 3 fonctions)

\codefrom{src}{encode}{encode_flux}

Passons maintenant au décodage d'un flux de booléens. Tout comme pour l'encodage il nous faut
une fonction préliminaire. Cette fois ci nous avons besoin de découper la liste de booléens
en entrée en sous listes de taille 32, ce qui nous amène à la définition suivante:

\codefrom{src}{encode}{cut_32}

Nous pouvons maintenant définir la fonction permettant de transformer un flux de booléens
en une liste d'instructions:

\codefrom{src}{encode}{decode_flux}



Nous pouvons maintenant définir les théorèmes finaux de notre assembleur qui sont les suivants:

\codefrom{src}{encode}{encode_decode_flux_decoup}

\codefrom{src}{encode}{decode_flux_decoup_encode}

\remark{Prouver ces deux théorèmes nécessite un travail assez conséquent au niveau des fonctions
  de découpage. Cependant ces preuves restant dans la même veine que celles que nous avons déjà
  vues précédement leur explication ne semble pas justifiée}






\section{Conclusion}
\label{conclusion}

Nous avons désormais un assembleur fonctionnel pour MMIX cependant celui- ci ne permet pas encore de se lancer
dans la réalisation de l'assembleur certifié x86.\\
La première raison qui me pousse à d'abord améliorer le parseur de MMIX est la représentation des données.
En effet la représentation présentée dans ce rapport est exhaustive car cela est encore possible avec
MMIX. D'une part l'architecture RISK limite le nombre de formats d'instructions. En x86 une instruction possède
beaucoup plus de champs et la taille de celle ci diffère selon les instructions. J'ai commencé à explorer certaines
pistes afin de factoriser MMIX que je ne présente pas dans ce rapport étant donné que celles- ci n'ont pas encore abouties.
\\
Une façon de résoudre ce problème de façon élégante et efficace serait de crée une librairie de parser.
En effet un assembleur s'attardant uniquement sur la syntaxe est un cas particulier de parser.
Grâce aux combinateurs de parseurs nous pourrions crée dynamiquement de nouveaux parser pour
permettre de parser l'ensemble des instructions avec un code factoriser. Le but serait donc de l'implémenter
dans un premier temps pour MMIX puis il faudrait adapter celui- ci pour suivre la spécification de x86.

\bibliographystyle{abbrvnat}
\bibliography{rapport.bib}  

\end{document}

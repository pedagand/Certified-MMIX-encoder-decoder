\documentclass {article}

%% ** Packages

\usepackage{natbib}
\usepackage[utf8]{inputenc}
\usepackage[french]{babel}
\usepackage{amsthm, amssymb, amsmath}
\usepackage{hyperref}
\usepackage{mathpartir}

%% ** OCaml listings

\usepackage{xcolor}
\usepackage{listings}
\lstset{
  language=[Objective]Caml,
  %% emph={[2]Nil,Cons,FZe,FSu,ze,su,Ze,Su},
  %% emphstyle={[2]\Constructor},
  %% emph={[3]lookup,failwith},
  %% emphstyle={[3]\Function},
  %% emph={[4]vec,fin,nat,list,tree,completeTree},
  %% emphstyle={[4]\Canonical},
  morecomment=[s]{(*}{*)},
  rangeprefix=\(\*\=,
  rangesuffix=\ \*\),
  includerangemarker=false,
  extendedchars=\true,
  inputencoding=utf8,
  showspaces=false,
  showstringspaces=false,
  showtabs=false,
  basicstyle=\ttfamily\small,
  framesep=4mm,
  moredelim=*[s][\itshape]{(*}{*)},
  moredelim=[is][\textcolor{darkgray}]{§}{§},
  escapechar=°,
  keywordstyle=\color[rgb]{0.627451, 0.125490, 0.941176},
  stringstyle=\color[rgb]{0.545098, 0.278431, 0.364706},
  commentstyle=\color[rgb]{0.698039, 0.133333, 0.133333},
  numberstyle=\color[rgb]{0.372549, 0.619608, 0.627451},
  boxpos=t,
  literate= {'a}{{$\alpha$}}1%
  {->}{{${\to}$}}2
  {*}{{${\times}$}}1
  {::}{{${:\::}$}}1
}

\newcommand{\codefrom}[3]
           {\lstinputlisting[linerange={#3}-End]{../#1/#2.v}}
           
%% ** Theorem styles 
           
           
\newtheorem{theorem}{Théorème}
\newtheorem{proposition}{Proposition}
\newtheorem{lemma}{Lemme}

\theoremstyle{definition}
\newtheorem{definition}{Définition}
\newtheorem{example}{Exemple}

\theoremstyle{remark}
\newtheorem{remark}{Remarque}
\newtheorem{para}{} 

%% ** commands

\newcommand{\todo}[1]{\textcolor{red}{#1}}
\newcommand{\attention}[1]{\textcolor{orange}{#1}}
\newcommand{\question}[1]{\textcolor{green}{#1}}
\newcommand{\etc}{\textit{etc.}}

\newenvironment{bnf}
               {\[\begin{array}{lcl@{\qquad}r}}
               {\end{array}\]}
               
\newcommand{\fun}[1]{\lstinline!#1!}
%% ** Title

\title{Assembleur x86 certifié}
\author{Roman Delgado}
\date{}


\begin{document}

\maketitle


%% ** Abstract 

\vfill
\setcounter{tocdepth}{2}
\tableofcontents
\vfill


\section{Introduction}

Les motivations ect...

\section{MMIX et représentation des données}


%TODO changer les affichages des essembles
\section{Conversion N liste B}

Code de bitn
\codefrom{src}{binary}{bitn}

Code de nbit
\codefrom{src}{binary}{nbit}


Parler du fait que pour ces lemme on a des effets de bords dont on est obligé
de faire des lemmes en prennant cela en compte 

Le but est de prouver un Lemme de cette forme
\codefrom{src}{binary}{nbitn}

et celui dans le sens opposé 
\codefrom{src}{binary}{bitnbit}






\section{Conversion Opcode liste B}


La première étape d'encodage ou de décodage d'une instruction est de


Afin de convertir l'ensemble opérateurs en binaire il nous faut
un moyens de stocker cette correspondance.
Ici le nombre d'étiquettes distinctes est de 256 ce qui est un nombre
assez faible. Une liste d'association se révèle ici une solution viable
étant donné que la recherche



Pourquoi ne pas utiliser de dictionnaire, parceque nous pourrons utiliser
la liste pour obtenir une le code binaire associé à une étiquette mais aussi faire
l'opération inverse


\section{Encode Decode}

\section{Conclusion}

\end{document}
               

\documentclass{beamer}

\usepackage[utf8]{inputenc}
\usepackage[francais]{babel}
\usepackage{amsthm, amssymb, amsmath}
\usepackage{mathpartir}
\usepackage{graphicx}
\usepackage{amsmath}
\usepackage{color}
\setlength{\fboxsep}{0pt}
\newcommand{\highlight}[1]{\text{\colorbox{gray}{$#1$}}}


\usepackage{xcolor}
\usepackage{listings}
\lstset{
  language=[Objective]Caml,
  %% emph={[2]Nil,Cons,FZe,FSu,ze,su,Ze,Su},
  %% emphstyle={[2]\Constructor},
  %% emph={[3]lookup,failwith},
  %% emphstyle={[3]\Function},
  %% emph={[4]vec,fin,nat,list,tree,completeTree},
  %% emphstyle={[4]\Canonical},
  morecomment=[s]{(*}{*)},
  rangeprefix=\(\*\=,
  rangesuffix=\ \*\),
  includerangemarker=false,
  extendedchars=\true,
  inputencoding=utf8,
  showspaces=false,
  showstringspaces=false,
  showtabs=false,
  basicstyle=\ttfamily,
  framesep=4mm,
  moredelim=*[s][\itshape]{(*}{*)},
  moredelim=[is][\textcolor{darkgray}]{§}{§},
  escapechar=°,
  keywordstyle=\color[rgb]{0.627451, 0.125490, 0.941176},
  stringstyle=\color[rgb]{0.545098, 0.278431, 0.364706},
  commentstyle=\color[rgb]{0.698039, 0.133333, 0.133333},
  numberstyle=\color[rgb]{0.372549, 0.619608, 0.627451},
  boxpos=t,
  literate= {'a}{{$\alpha$}}1%
    {->}{{${\to}$}}2
    {*}{{${\times}$}}1
    {::}{{${:\::}$}}1
    {lambda}{{$\lambda$}}1
}

\newcommand{\codefrom}[3]
           {\lstinputlisting[linerange={#3}-End]{../#1/#2.v}}
           \newcommand{\fun}[1]{\lstinline!#1!}
\newcommand{\codefromOcaml}[3]
           {\lstinputlisting[linerange={#3}-End]{../#1/#2.ml}}

           
\newcommand{\intg}{\ensuremath{\mathsf{int}}}
\newcommand{\bool}{\ensuremath{\mathsf{bool}}}
\newcommand{\Lam}[2]{\ensuremath{\lambda #1\: #2}}
\newcommand{\App}[2]{\ensuremath{#1\:#2}}
\newcommand{\Var}[1]{\ensuremath{#1}}
\newcommand{\Fst}[1]{\ensuremath{#1.\pi_0}}
\newcommand{\Snd}[1]{\ensuremath{#1.\pi_1}}
\newcommand{\Pair}[2]{\ensuremath{(#1, #2)}}
\newcommand{\ifte}[4]{\ensuremath{\mathsf{ifte}\:#1\: #2\:\:#3\:\: #4}}
\newcommand{\true}{\ensuremath{\mathsf{true}}}
\newcommand{\false}{\ensuremath{\mathsf{false}}}
\newcommand{\zero}{\ensuremath{\mathsf{zero}}} 
\newcommand{\succs}{\ensuremath{\mathsf{succ}}}
\newcommand{\iter}{\ensuremath{\mathsf{iter}}}
\newcommand{\subst}[3]{#1[#2 := #3]}
\newcommand{\Inv}[1]{\ensuremath{\mathsf{inv}(#1)}}
\newcommand{\Ann}[2]{\ensuremath{(#1\: :\: #2)}}
\newcommand{\equal}[3]{\ensuremath{#1 =_#2 #3}} 
\newcommand{\refl}{\ensuremath{\mathsf{refl}}}
\newcommand{\id}[3]{\ensuremath{\mathsf{id}\:#1\: #2\:\:#3}}
\newcommand{\vect}[2]{\ensuremath{\mathsf{vec}\:#1\: #2}}
\newcommand{\dfold}[6]{\ensuremath{\mathsf{fold}\:#1\:#2\:#3\:#4\:#5\:#6}}
\newcommand{\cons}[2]{\ensuremath{\mathsf{cons}\:#1\: #2}}


\setbeamertemplate{navigation symbols}{} 

\usetheme{Boadilla}

\title{Assembleur Certifié}

\author{Roman Delgado}

\institute[\textsc{Upmc}]{Université Pierre et Marie Curie}

\date{31/05/2017}


\begin{document}


\begin{frame}

\titlepage

\end{frame}

%%%%%%%%%%%%%%%%%%%%%%%%%%%%%%%%%%%%%%%%%%%%%%%%%%%%%%%%%%%%%%%%%%%%

\begin{frame}
  \frametitle{Introduction}

  \begin{block}{Objectifs du projet}
    \begin{itemize}
    \item Apprentissage du langage Coq
    \item Réaliser un assembleur certifié
    \end{itemize}
  \end{block}

  \begin{block}{Pourquoi certifier un assembleur}
    \begin{itemize}
    \item Bugs KVM (décodage des instructions)
    \item Fin de la chaîne de compilation 
    \end{itemize}
  \end{block}   
  %surtout au niveau du decodage que c'est la galère , on ne peut pas vraiment allez plus loins que ça après c'est le matériel
\end{frame}
%%%%%%%%%%%%%%%%%%%%%%%%%%%%%%%%%%%%%%%%%%%%%%%%%%%%%%%%%%%%%%%%%%%%
\begin{frame}[b,fragile]

\frametitle{Programmes certifiés}

\vfill

\begin{block}{Qu'est-ce qu'un programme certifié?}
  \begin{itemize}
  \item Une spécification formelle
  \item Une preuve que le programme respecte sa spécification
  \end{itemize}
\end{block}
\begin{block}{Programmation certifiée dans Coq}
  \begin{itemize}
  \item Coq est un assistant de preuve
  \item Programme écrit avec Coq 
  \item Preuves également vérifiées par Coq
  \item Programme exécuté = programme prouvé      
  \end{itemize}
\end{block}
\vfill


\end{frame}


%%%%%%%%%%%%%%%%%%%%%%%%%%%%%%%%%%%%%%%%%%%%%%%%%%%%%%%%%%%%%%%%%%%%
\begin{frame}
\frametitle{Certifier un Assembleur}

%%L'assembleur est un traducteur entre un langage cible et un langage source Pour mmix on peut voir le fait de faire un assembleur
%%comme définir une relation isomorphique entre les deux ensembles alors que pour x86 c'est pas vrais
\includegraphics[scale=0.8]{modif.png}

%je met pas les mêmes parceque elles ne possèdent pas le même ensemble de définition
\begin{block}{Fonctions Inverses}
\begin{itemize}
\item Decodage(Encodage(y)) = y
\item Encodage(Decodage(x)) = x  \hspace{2cm} $\forall x \in Def(Decodage)$
\end{itemize}  
\end{block}

\end{frame}

%%%%%%%%%%%%%%%%%%%%%%%%%%%%%%%%%%%%%%%%%%%%%%%%%%%%%%%%%%%%%%%%%%%%
\begin{frame}
\frametitle{x86 vs MMIX}

\begin{block}{x86}
  \begin{itemize}
    \item Taille des instructions variable
    \item Une même instruction possède différents encodages
    \item De nombreux champs dans une instruction (tag, ect.)
    \end{itemize}  
\end{block}

\begin{block}{MMIX}
  \begin{itemize}
  \item Jeu d'instructions RISC (taille des instructions fixée)
  \item Format des instructions très simple
  \end{itemize}  
\end{block}

\vfill

\end{frame}
%%%%%%%%%%%%%%%%%%%%%%%%%%%%%%%%%%%%%%%%%%%%%%%%%%%%%%%%%%%%%%%%%%%%
\begin{frame}
\frametitle{Partialité} 

\begin{block}{Monade option}
  \begin{itemize}
  \item Mécanisme pour gérer les fonctions partielles
    %C'est un type paramétrique 
  \item \fontsize{8}{10} \codefrom{rapport}{definitions}{option}
  \end{itemize}  
\end{block}
\begin{block}{Exemple fonction de décodage}
  \fontsize{8}{10} \codefrom{src}{encode}{decode}
\end{block}

\begin{block}{Impact sur les théorèmes}
  \fontsize{8}{10} \codefrom{src}{decodeProof}{decode_encode}
\end{block}
  




\end{frame}


%%%%%%%%%%%%%%%%%%%%%%%%%%%%%%%%%%%%%%%%%%%%%%%%%%%%%%%%%%%%%%%%%%%%
\begin{frame}
\frametitle{Récursion} 

\begin{block}{Exemple Ocaml}
  \fontsize{8}{10} \codefromOcaml{rapport}{example}{nbit}
\end{block}


\begin{block}{Exemple Coq}
  \fontsize{8}{10} \codefrom{src}{binary}{nbit}
\end{block}


\end{frame}


%%%%%%%%%%%%%%%%%%%%%%%%%%%%%%%%%%%%%%%%%%%%%%%%%%%%%%%%%%%%%%%%%%%%
\begin{frame}
\frametitle{Réflexion} 

\begin{block}{Théorèmes}
  \begin{itemize}
  \item \fontsize{8}{10} \codefrom{src}{association_list}{lookup_lookdown}
  \item \fontsize{8}{10} \codefrom{src}{association_list}{lookdown_lookup}
  \end{itemize}    
\end{block}


\begin{block}{Avantages}
  \begin{itemize}
  \item Terme de preuve plus petit
  \item Plus d'études de cas pour chaque nouveau théorème
  \end{itemize}    
\end{block}

\end{frame}


%%%%%%%%%%%%%%%%%%%%%%%%%%%%%%%%%%%%%%%%%%%%%%%%%%%%%%%%%%%%%%%%%%%%
\begin{frame}
  \frametitle{Conclusion}
  
  \begin{block}{Réalisations}
    \begin{itemize}
    \item Conversion \ensuremath{\mathbb{N}} vers liste \ensuremath{\mathbb{B}}
    \item Fonctions d'encodage et de décodage
    \item Encodage de flux d'instructions et décodage d'une suite de bits
    \end{itemize}
  \end{block}  
  
  \begin{block}{Vers un assembleur x86}
    \begin{itemize}
    \item Représentation des données
      % Comme détaillé dans le rapport pour rendre plus aisé les preuves j'ai préférer utiliser une représentation des données la plus simple possible cependant cela induit des redondances dans certaines preuves. Etant donné le trop grand nombre de syntaxes différentes pour les instructions en x86 il 'est pas possible de faire la même chose
    \item Rendre le programme actuel plus compositionnel
      %argument c'est parceque il y a trop de cas particuliers dans x86 comparé à MMIX
    \end{itemize}
  \end{block}
  
\end{frame}







\end{document}
